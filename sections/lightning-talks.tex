%**************************************************************************
\section{Lightning Talks}\label{sec:lightning-talks}
%**************************************************************************

Participants were also encouraged to volunteer for a lightning talk to provide
a perspective into their recent measurement research work.

% -------- Renata
%\subsection{HostView: Measuring Internet quality of experience on end-hosts}
\subsection{HostView}

There is interest in automated performance diagnosis on user laptops or
desktops. One interesting aspect that has received little attention is the
user perspective on performance. To conduct research on both end-host
performance diagnosis and user perception of network and application
performance, Renata Teixeira (INRIA) presented an end-host data collection
tool, called HostView. HostView \cite{rteixeira:pam:2012} not only collects
network, application and machine level data, but also gathers feedback
directly from users. User feedback is obtained via two mechanisms: a
system-triggered questionnaire and a user-triggered feedback form. In her
talk, she described experiences with the first deployment of HostView. Using
data from 40 users, she articulated challenges in this line of research, and
reported initial findings in correlating user data to system-level data. She
then described more recent efforts in conducting an in-depth study with 12
users in France to guide the design of the next version of HostView and of
methods to infer user context and activities.

% -------- Al Morton:
\subsection{Virtual Measurement Accuracy}

The movement towards \ac{NFV} means that measurement system virtualisation
will take place for active, passive, and hybrid methods of measurement. This
evolution will allow on-demand deployment of measurement systems in general
purpose servers. The designs must be cost-effective, but there is tension
between cost of physical resources and accuracy.  Al Morton (AT\&T) presented
this trade-off and associated challenges.

% -------- Brian Trammell:
\subsection{A Path Transparency Observatory}

The growing deployment of middle boxes in the Internet has reduced the degree
to which the Internet is still an end-to-end network in accordance with its
original design. This lack of end-to-endness leads to ossification of the
transport layer \cite{mhonda:imc:2011}: new protocols are difficult or
impossible to deploy as they must be designed around middle boxes, either
those which have been observed, or conjectured to exist. It is necessary to
guide protocol engineering for transport protocol innovation on a basis of
observations of the Internet as it is, but these observations are hard to come
by. Brian Trammell (ETH Z\"urich) proposed a Path Transparency Observatory,
which can take observations of path transparency (the likelihood a packet
stream that arrives at the end of the path is the one that was sent, with
certain properties) and impairment (something that keeps a path from being
transparent for a certain kind of traffic) from multiple sources, with
multiple resolutions of condition definition and information about the
endpoints and path involved. An observatory collects single observations of a
path and a condition on that path at some point in time, with references to
the code that created the observations so they can be repeated, and a set of
equivalence functions so that equivalent conditions and paths can be compared.
He explained that this work is ongoing, and a public observatory will become
available within the scope of the \ac{MAMI} project \cite{mami} over next two
and a half years.

% -------- Varun Singh
%\subsection{One year of measuring WebRTC service quality in the Wild}
\subsection{WebRTC Service Quality in the Wild}

Varun Singh (Aalto University) introduced callstats.io, a \ac{WebRTC}
analytics and diagnostics service. It measures service- and conference-level
metrics for a \ac{WebRTC} application service. At the service-level,
annoyances (such as, how often do conferences fail, what are the reasons for
failure and what is the typical network latency?) are measured. Varun
described how callstats.io will share the aggregate quality metrics measured
across tens of \ac{WebRTC} services (big and small, local and global) with the
measurement community at large.

% -------- Georg Carle:
\subsection{From Local to Global Measurements}

Georg Carle (TU M\"unchen) provided a summary of measurement based research
work conducted within his group. He explained that understanding Internet
phenomena requires both local and global measurements. Local measurements such
as on the MEMPHIS test bed allows for reproducible experiments. As part of
this project, the MoonGen Traffic generator \cite{gcarle:imc:2015} is an
example that allows for high precision by directly accessing hardware features
such as precise time stamping from the application space, while bypassing the
operating system.  Furthermore, Georg reasoned how \ac{SDN} mechanisms can be
used for performing very high-speed flow monitoring using \ac{COTS} components
and adaptive load-balancing. One objective of security-related global
measurements is to identify prefix hijacking. An innovative approach
\cite{gcarle:tma:2015} to identify benign anomalies is to use information with
business relations and ownership information from a publicly accessible
\ac{IRR} to combine it with collected TLS certificates, and using these
certificates as fixed points to be checked in time intervals in which routing
anomalies are observed. For performing measurements with wireless links, he
presented the MeasrDroid Android app, which allows to perform wireless
measurements from many vantage points.

% -------- Markus Fiedler
\subsection{From Packet Counts to QoE}

Markus Fiedler (BTH) discussed challenges in measuring \ac{QoE}. He stressed
that the main challenge for the interpretation of network measurements in
light of \ac{QoE} is that user perception happens far up the network stack,
far away from where \ac{QoS} problems (such as latency and packet loss) arise
and where monitoring takes place.  \ac{QoS} may be transformed significantly
throughout the stack. The recently proposed \ac{QoE} Hourglass Model
\cite{mfiedler:commantel:2013} is one way to formalize such transformations,
capturing impacts of transport protocols, display devices and other factors.
Using an example from a project with a major European telecommunications
provider, he proposed a method that enables the exploitation of information
from packet counters for \ac{QoE} assessment.  It starts with the definition
of the user-perceived problem to be attacked, followed by the determination of
parameters that reflect those problems far down in the network stack, and of
the critical timescale for the user, and finally the use of appropriate
comparative summary statistics.

% -------- Ian Marsh
%\subsection{CheesePi - Swedish home network monitoring}
\subsection{CheesePi}

Ian Marsh (SICS) described the architecture of a distributed measurement
system, CheesePi. He utilized the IETF LMAP framework \cite{rfc7594}
terminology to describe this system.  CheesePi uses Raspberry Pi hardware
devices to allow always-on, simple and reliable monitoring of users’ home
Internet connections. By running CheesePi on a Raspberry Pi (termed a \ac{MA})
connected to their home network, a non-expert user can continuously monitor
their connection quality. He argued that a common hardware platform for all
\ac{MA}s gives greater consistency between the collected measurements and it
also simplifies the codebase. The result is a common software platform for
measurement tasks that can host, execute and record arbitrary network
behaviour. Ian explained the reason for deploying dedicated monitoring
devices. The project is tailored towards capturing the network connectivity
that devices are able to achieve. This can significantly depend on the last
hop technology (e.g. Ethernet or WiFi), which would be missed by passive
monitoring of user traffic at the home gateway.  This work is performed in
collaboration with the Swedish regulator \ac{PTS}, who are particularly
concerned with expanding connection performance metrics from naive throughput
measurements of a particular location and time to something more instructive.
He argued that an easily comparable and widely understood metric (e.g.,
download/upload rates) does not necessarily indicate the \ac{QoE} of a user.

% -------- Burkhard Stiller:
%\subsection{Schengen Routing: A Compliance Analysis}
\subsection{Schengen Routing}

Burkhard Stiller (UZH) described Schengen routing as a strategy to keep
traffic originating from sources located in the Schengen area (an area
comprising of 26 European countries that have abolished passport and any other
type of border control at their common borders)  and targeted to destinations
located in the Schengen area within the Schengen area. He summarised results
of a larger-scale measurement effort \cite{bstiller:aims:2015} performed to
quantify Schengen routing compliance in parts of today's Internet. Based on a
few thousand TCP, UDP, and ICMP \texttt{traceroute} measurements executed from
RIPE Atlas probes located in over 1100 different \ac{AS} in the Schengen area,
it was observed that 34.5\% to 39.7\% of these routes are Schengen-compliant,
while compliance levels vary from 0\% to 80\% among countries.

% -------- Srikanth and/or Phillipa
%\subsection{Haystack - Mobile traffic monitoring in user space}
\subsection{Haystack}

Despite our growing reliance on mobile phones for a wide range of daily tasks,
we remain largely in the dark about the operation and performance of our
devices, including how (or whether) they protect the information we entrust to
them, and with whom they share it. The absence of easy, device-local access to
the traffic of our mobile phones presents a fundamental impediment to
improving this state of affairs. To develop detailed visibility, Srikanth
Sundaresan (ICSI) presented Haystack \cite{ssundaresan:arxiv:2015}, a system
for unobtrusive and comprehensive monitoring of network communications on
mobile phones, entirely from user-space. Haystack correlates disparate
contextual information such as app identifiers and radio state with specific
traffic flows destined to remote services, even if encrypted.  Haystack
facilitates user-friendly, large-scale deployment of mobile traffic
measurements and services to illuminate mobile app performance, privacy and
security. Srikanth described the design of Haystack and demonstrated its
feasibility with an implementation that provides 26-55 Mbps throughput with
less than 5\% CPU overhead. He stressed that the system and results highlight
the potential for client-side traffic analysis to help understand the mobile
ecosystem at scale.

