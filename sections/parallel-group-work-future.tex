% --------- 1. Measurement method evolution

Measurement methods will evolve beyond traditional active and passive
techniques. Al Morton in \cite{draft-ietf-ippm-active-passive} describes
hybrid measurement methods which are subset of both active and passive
methods. For instance, Type I hybrid measurements employ methods that augment
or modify the stream of interest, while Type II hybrid measurements employ
methods that utilize two or more streams of interest with some degree of
mutual coordination to collect multiple metrics.

% --------- 2. Measurements involving wireless links

A number of Internet measurement tools are designed with inherent assumptions
(about layer-2 networks) \cite{ssundaresan:pam:2015} that are not true for
underlying wireless links.  In particular WiFi home networks and cellular
networks are impacted by bitrates, retransmission rates, and signal strengths
as wireless channel conditions change. As such, we need to design measurement
approaches and tools that are also suitable for measuring wireless links.

% --------- 3. Challenging measurement metric available bandwidth

There are also challenges with metrics that measure available bandwidth.  In
the view of mostly elastic traffic, partly in combination with wireless links,
it is not clear whether a convincing solution can be expected. Moreover, with
existing tools, probing for capacity does not work well with tools that assume
that the link is work-conserving.

% --------- 4. Traffic mixes at different parts of the network

For many web-based applications, end-to-end traffic is split
\cite{apathak:pam:2010} into a transport session from end system to the
front-end servers, and another transport session to the backend
infrastructure.  In the transport session to the front-end servers, many
short-term TCP flows may be observed (in contrast to long-lived TCP flows in
the transport session to the backend infrastructure). In such a scenario,
protocols used to establish the transport session to the front-end servers can
be changed quickly. For instance, \ac{QUIC} \cite{draft-tsvwg-quic-protocol},
which is increasingly used to establish a transport session to the front-end
servers, may behave more aggressively than TCP.

% ---------- 5. Measurements of Low Latency Services
% ---------- 7. Internet of Things (IoT) Measurements

There is an increasing demand for low-latency communication.  Many
technological advances reduce latency significantly. For instance, compared
with 4G, 5G claims that it will reduce latency by a factor of 100. However, it
is unclear how one can measure latency in these new environments with the
required level of accuracy. The security aspects of \ac{IoT} devices are
becoming critical. It is unclear whether there is a need for specialized
measurement tools and methods in this space. An analysis of measurement
challenges with respect to \ac{IoT} security is needed.

% ---------- 6. Network Function Virtualisation Measurements

A large number of network functions are being virtualised today.  It remains
unclear how to measure in such virtualised scenarios. Additional measurement
objectives and metrics need to be identified particularly due to the resource
sharing effects of such virtual network functions.
