%**************************************************************************
\section{Introduction}\label{sec:introduction}
%**************************************************************************

%------------------------ Motivation

Several large-scale Internet measurement platforms have been deployed during
the last years in order to understand how the Internet is performing, to
observe how it is evolving, and to determine where failures or degradations
occur. Examples are the CAIDA \ac{Ark} platform \cite{kclaffy:ccr:2016} (used
for Internet topology discovery and detecting congestion on interdomain
links), the SamKnows platform \cite{vbajpai:comst:2015} (used by regulators
and network operators to study network performance), the RIPE Atlas platform
\cite{ripencc:ipj:2015, vbajpai:ccr:2015} (that provides measurement services
to network operators and researchers), the Netradar system
\cite{ssonntag:wiopt:2013} (for performing wireless performance measurements),
and the BISmark project \cite{ssundaresan:usenix:2014}.  European
collaborative research projects lately have been working on a \ac{mPlane}
\cite{btrammell:commag:2014} and how to incorporate measurement results into
network management systems (e.g., Leone) \cite{leone}. Related projects (e.g.,
Flamingo) \cite{flamingo} are increasingly working with measurement data from
these platforms.  Large-scale measurements are meanwhile also used to drive
network operations or to dynamically adjust how services are delivered to
customers. \ac{CDN} providers use measurement data to optimize content caches
and to tune load balancing algorithms. One key challenge is that global
Internet measurement systems can generate large amounts of data that need to
be processed to derive relevant information.

%------------------------ Goals

This seminar (\#16012) was a followup of the Dagstuhl seminar on Global
Measurement Frameworks (\#13472) \cite{peardley:dagstuhl:2013}. The main focus
of the first seminar was an exchange of ideas on the development of global
measurement infrastructures, frameworks and associated metrics. Some of this
work is now further pursued in standardization bodies
\cite{vbajpai:comst:2015}  such as the IETF \ac{LMAP} working group and the
Broadband Forum.  The goal of this followup seminar was to focus on the
experience obtained with different metrics, tools, and data analysis
techniques. It provided a forum for researchers to exchange their experience
with different practices to conduct global measurements. The aim was to
identify what works well in certain contexts, what has proven problematic in
other contexts, and identify open issues that need further research.

The seminar approached this by looking at three distinct dimensions: $a)$
Measurement metrics, $b)$ data processing technologies and $c)$ data analysis
methodologies. Some key questions were: $a)$ Which metrics have been found
useful for measuring \ac{QoE} of certain classes of services?  Which metrics
have been found problematic? Is it possible to find indicators for good
metrics and problematic metrics?, $b)$ Which technologies have been found
useful for storing and processing large amounts of measurement data?  Which
technologies were found to be problematic?  Are there new promising
technologies that may be used in the future? What are the specific
requirements for dealing with large-scale measurement data and how do they
relate to or differ from other big data applications? and $c)$ Which data
analysis techniques have been found to be useful? Which data analysis
techniques have been found to be problematic? Are there any novel promising
techniques that need further research and development?

Although at the seminar the participants chose to organize the discussions on
more general topics than these specific questions, during the discussions most
of these questions were addressed to one degree or another.
