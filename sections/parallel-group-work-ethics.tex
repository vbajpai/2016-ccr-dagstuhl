There can be a tension between scientific principles (measurements and
meta-data should be public) and consistency with ethical principles.  For
instance, we have recently witnessed controversial papers
\cite{sburnett:sigcomm:2015, mdischinger:imc:2007} that, although published,
raised ethical concerns within the program committee.

There has already been activity by the community on ethical practice. For
instance, a dedicated SIGCOMM workshop on Ethics in Networked Systems Research
\cite{ethics} was recently organized in 2015. Moreover, the call for papers
for \ac{IMC} encourages authors when appropriate to include a subsection
describing ethical considerations and provides appropriate links for further
information on ethical principles \cite{menloreport} and guidance
\cite{mallman:imc:2007} on ethical data sharing. As a followup to the Dagstuhl
seminar on Ethics in Data Sharing \cite{jcohen:dagstuhl:2014,
roland:creds:2014}, SURFnet is preparing a document \cite{roland:tnc:2015} on
Data Sharing policy.  The final policy will most likely come into effect in
the first quarter of 2016.

The issue of what can be considered ethical is often a grey area since
opinions can dramatically vary by different parties. For example, a study
\cite{anonymous:ccr:2012} that analyses causes of collateral damage of
censorship by identifying DNS injection activities of the Great Firewall of
China could potentially be viewed as unethical by the government of the
People's Republic of China.  Several intriguing questions from the ethical
standpoint deserve discussion.  For instance, in a measurement study of cyber
crime, is it appropriate to buy products from criminals? and is it appropriate
to crawl a website to obtain all of the information even when the site
explicitly states that one should not do this?

%-------- Some additional observations:

Ethical issues pertain to more than just privacy infringement. For instance,
disrupting the service of an end-user and possibly even endangering an
end-user without the user's consent. There is a fine line between legal and
ethical issues. Moreover, ethical issues encompass the entire measurement
chain starting from the design of an experiment, conducting measurements, data
storage, data processing, and data sharing. The security research community is
increasingly sensitive to this issue. For instance, some IT departments avoid
collecting data just so that they have no data if asked by a law-enforcement
agency. By nature, research tends to push the boundaries, however risk
analysis can be hard. If it is known in advance how the data will be used,
collect just what is needed.

%------- What is to be done.

The Internet measurement community needs to publish further guidelines on
ethical practice. A key target audience is researchers that are not aware of
the issues, but would want to do the right thing. For the Internet measurement
community, continued discussion to gain more clarity in the aforementioned
grey areas is needed.  Regardless of whether there is consensus in the
community as a whole, a conference program committee should have the
discretion to reject a paper on ethical grounds. The authors of the rejected
paper could be asked for permission to make known the aspect of the work that
was considered unethical, so as to provide guidance to the wider community.
Furthermore, since a paper rejection occurs after the unethical practice has
already occurred, the goal must be to avoid the unethical practice to happen
in the first place. To address this part, an interesting question is how to
have curricula embrace ethics educations. An ethics background is not just
only needed for measurement studies, but in general for people working in
computer science, both in academia and industry.

%--- Useful links:
%http://www.ethicalresearch.org/ -- project by Stuart Schecter and others.
%http://www.ethicalresearch.org/efp/netsec/ -- Ethics Feedback Panel for Networking and Security
%http://www.oii.ox.ac.uk/people/?id=281 -- Ben Zevenbergen: doing lots of good work in this area. His project page: http://ensr.oii.ox.ac.uk/
